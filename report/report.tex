\documentclass{article}
\usepackage[utf8]{inputenc}
\usepackage{amsmath}
\usepackage{enumitem}
\usepackage{graphicx}
\usepackage{hyperref}



\title{ THE TITLE \\
Service task - report \& notes}
\author{Sebastian Bysiak}
\date{December 2018 -- ...}

\graphicspath{ {images/} }




\begin{document}
\maketitle


\section{Preliminary remarks}

\textit{
This document is created for keeping track of the progress, major and minor subtasks and milestones related to my service task.\\
It should contain results of all related analyses but also notes of various thing I learnt during preparing and working on the service task.
Basically every day spent on the project should result in some figure or note in this document, except for whole day of dump debugging. 
Therefore notes should be brief and compact with sources/references where applicable without messing up.
}


\section{Notes - varia}

\subsubsection{PCA \& linear autoencoders}
Linear autoencoders (AE) searches same space of the solutions - they can yield same result.
Linear AE could be useful in some cases like: 
\begin{itemize}[topsep=0pt, itemsep=-2pt]
    %\setlength\itemsep{-0.5em}
	\item[-] large dataset (hard to fit in memory) - there is also incremental PCA (with sklearn implementation)
	\item[-] online learning 
\end{itemize}

Robust PCA = PCA + sensitivity to anomalies
splits data into low-rank matrix L and sparse matrix S.
L is low dim approximation of non-anomaly examples \vspace{0pt} \\
\href{https://www.youtube.com/watch?v=eFQVvFMHlC8}{\texttt{youtube.com}} -- 3min, "Anomaly Detection with Robust Deep Auto-encoders"\\
\url{https://github.com/dlaptev/RobustPCA} 







\end{document}
